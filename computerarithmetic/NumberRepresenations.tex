\chapter{Number Representations}

\section{Dot notation}
\paragraph{}Dot notation Concepts 
\begin{itemize}
	\item Posibit\quad  \(\newmoon\quad\in\quad [0,1]\) 
	\item Negibit\quad	\(\ocircle\quad\in\quad [-1,0]\)
\end{itemize}

It is the most human and error-free type of Number Representations\\
\\
Multiplication for unsigned numbers:
\begin{align*}
	\newmoon\quad\newmoon\quad\newmoon \\
	\mathcal{\times}\quad\newmoon\quad\newmoon\quad\newmoon\\
	\--------\\
	\newmoon\quad\newmoon\quad\newmoon\quad\newmoon\\
	\newmoon\quad\newmoon\quad\newmoon\quad\newmoon\quad\quad\\
	\newmoon\quad\newmoon\quad\newmoon\quad\newmoon\qquad\qquad\\
	\newmoon\quad\newmoon\quad\newmoon\quad\newmoon\qquad\qquad\qquad\\
	\--\----\-----------\\
	\newmoon\quad\newmoon\quad\newmoon\quad\newmoon\quad\newmoon\quad\newmoon\quad\newmoon\quad\newmoon
\end{align*}\\
Addition for unsigned numbers:
\begin{align*}
	\newmoon\quad\newmoon\quad\newmoon \\
	\textbf{+}\quad\newmoon\quad\newmoon\quad\newmoon\\
	\--------\\
	\quad\newmoon\quad\newmoon\quad\newmoon\quad\newmoon
\end{align*}\\
\\
There are many other number representations but the most important ones are:
\begin{enumerate}
	\item \textbf{2's} Complement
	\item \textbf{Binary Stored-carry} or \textbf{Carry-saved} format
	\item \textbf{Binary floating point number } (IEEE 754)
	\item \textbf{BCD}
\end{enumerate}

\section{Signed Number Representations}
\subsection{Signed-Magnitude Representation}
Definition: The \textbf{most left bit} is the \textbf{sign bit} (s) \[
	\mathbf{if}
\left\{\begin{array}{cl}
	s\,=\,0 & \Longrightarrow \text{positive number}\\s\,=1 & \Longrightarrow \text{negative number} 
\end{array}\right\}
\]
Numbers of this Representation type are \textit{fix-point and with no fraction}\\
In \textit{Radix r} the number \(k\)  of digits needed for representing [0,max] is
\[
	k\,=\,\left\lfloor\,log_{r}max\,+\,1 \right\rfloor\,+1\,=\,\left\lceil log_{r}(max+1) \right\rceil 
\]
\\
Example: for Radix=2 and range is [0,7] how many digits are needed?
\\\\Solution: \(k\,=\left\lceil log_{2}(7+1) \right\rceil \Longrightarrow\, 3\) so there are 3 digits are needed for representing this digit set\\\\
\textbf{Disadvantages of Signed-Magnitude Representation}
\begin{enumerate}
	\item Because of Symmetric nature of this representation there will be two \(0\)s with different signs \(\pm0\). This is unavoidable in Radix-2 Symmetric Systems.
	\item More overhead and thus more delay because of \(\pm0\) existence.
\end{enumerate}
Finally the Digit set of \textbf{Signed-Magnitude Representation} is 
\[
	\big[-(2^{k-1}\,-\,1)\,,2^{k\,-\,1}\,-1\big]
\]


\subsection{Biased Representations}
\paragraph{}
As the name suggests\textit{\textbf{ a bias}} is applied to the \textbf{signed number}; which then can be used for conversion from\textbf{ Signed} to \textbf{Unsigned} numbers.\\
The\textbf{ Digit set }can be shown as \( \mathbf{{[-bias,max-bias}]} \); using values from \textbf{0} to \textbf{max}, this is often called \textbf{"excess-bias"}. Most notable examples are \textbf{"excess-3" or (BCD)} and \textbf{"excess-128"} (is used for simpler hardware) coding.
Formula Demonstration:\begin{align*}
	x+y-bias=(x+bias)+(y+bias)-bias \\
	x-y+bias=(x+bias)-(y+bias)+bias\\
	\text{with kbit numbers and  a bias of } \mathbf{2^{k-1}}
\end{align*}
The Complexity is\textbf{ negligible} for this type.\\
Multiplications and Divisions become \textit{more difficult} if applied on \textbf{biased numbers}, thus the \textit{bias representation} is mostly suited for the \textbf{exponent (e)} part of \textbf{the floating-point numbers}, since they are \underline{\textbf{never}} multiplied or divided. A common example of this is \textbf{IEEE 754 standard}.