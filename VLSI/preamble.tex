\section{VLSI Design Problems}
\paragraph{}
\textbf{Definition:}\vspace{1em}\\
{\textbf{VLSI}} is the Antonym of \textit{Very Large Scale Integration} \cite{lsi} . In general an \textbf{IC} or \textbf{technology} that contain more than \(\mathbf{10^{5}}\) transistors are considered \textbf{VLSI}. There is also \textbf{LSI} (Large Scale Integration); albeit having different spelling, they are used interchangeably. There are \textbf{six} crucial factors in VLSI Optimization.\\
\begin{enumerate}
\item Area
\item Speed
\item Power Consumption 
\item Thermal Management
\item Signal Integrity
\item Congestion
\end{enumerate}

The problem is one can not simply improve all the factors, there are trade-offs between these factors. Also \textbf{Design Time} is an important thing to consider in the economical context, it is worth mentioning that\textbf{ Testability} is an important factor to consider; because a noticeable percentage of the fabricated chips are defective, this makes testing chips a vital process before being used in the product. \\ VLSI is can be a\textbf{ general-purpose integrated circuit}(e.g. microprocessor) or an \textbf{application-specific integrated circuits(ASICs)}.
Relations between the main factor can be expressed as:\[
	\mathit{Speed}\,\alpha\,\frac{1}{\mathit{Area}}\quad \,\mathit{Power\,Consumption}\,\alpha\,\frac{1}{\mathit{Area}} 
\]
The smaller \textbf{Area} means \textbf{less silicon }consumption and increase in the\textbf{ yield}. \textit{The Larger the Area the higher chance of defects}.\\
\textbf{Speed} is a\textbf{ design constraint}, meaning optimization revolves around the designated \textbf{Speed}. 