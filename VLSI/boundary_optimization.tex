\section{Boundary Optimization}
\paragraph{}
Definition: In simple terms \emph{Boundary Optimization} can be defined as the set of techniques
aiming for the improvement of \textbf{circuits} both in \textbf{efficiency and performance} at the boundaries between \textbf{different logic modules}
	
\paragraph{}
\textbf{Types of Boundary Optimization} \\
There are \emph{four} main types of \textbf{Boundary Optimization} 
\begin{enumerate}[label=\textcolor{blue}{\arabic*.}]
\item \textcolor{blue}{\textbf{Boundary Condition Optimization}} 
\item \textcolor{blue}{\textbf{Boundary Merging}}
\item \textcolor{blue}{\textbf{Boundary Logic Optimization}}
\item \textcolor{blue}{\textbf{Retiming}}
\end{enumerate}
\paragraph{}
\textbf{Boundary Condition Optimization} \\
This optimization is focused on improving \emph{signal integrity} while crossing from one module to another. The main benefits are:
\begin{enumerate}
\item Preventing Timing Violations
\item Better Timing Performance 
\item Power Consumption Reduction 
\end{enumerate}
\paragraph{}
\textbf{Boundary Merging} \\
This Techniques focuses on merging \textbf{adjacent blocks or modules} with \underline{similar} functionality. This techniques has several benefits:
\begin{enumerate}
\item Area Reduction
\item Reduction in Power Consumption
\item Enhance Timing
\end{enumerate}
\paragraph{}
{\textbf{Boundary Logic Optimization}} \\
As the name suggests, the technique is focused on logic redundancy elimination, thus \textbf{simplifying the logic equation complexity}. In Addition it \textbf{simplifies the design} and \textbf{Reduces the power consumption}.
\paragraph{}
\textbf{Re-timing} \linebreak
The mechanism behind \textbf{Re-timing} is based on \textit{flip-flops and registers} re-positioning; The \textbf{clock skew will be reduced} and \textbf{bottlenecks will be prevented}.