\chapter{Preliminary}
\section{Radix and Base Conversion}
\paragraph{}
Digits are represented based on their \textbf{Radix}.\\
The most commonly use Radixes are \(2,\,8,\,10,\,16\), that are called Binary, Octal, Decimal and, Hexadecimal respectively.\\
A number (N) with raidx (r) can be written as\large{\[
(N)_{r} = d_{n-1} d_{n-2}\cdots\,\,d_{1}\,d_{0} . d_{-1} d_{-2} \cdots d_{-m}
\]}
where \(d_{n-1}\) is the MSB and \(d_{-m}\) is the LSB. The negative indexes are the fractional part.\\
The General Radix Conversion rule from Radix 10 to Radix (R) can be described as \[
\left(d_{n-1}\dots d_{1}d_{0}=\sum _{i=0}^{n-1}d_{i}\cdot R^{i}\right)
\]\label{conv}
It goes without saying that the conversion from R to decimal(Radix=10) is the inverted form of the formula.
\paragraph{}
There are three ways of showing a negative number. The first one is to set the \textit{MSB} if the number is negative, for example -2 with Radix of 2 can be represented as \((10000010)_{2}\) in bits. However this representation is not a proper choice because there will be twp \textit{0s}, one with positive sign and the other with negative sign.\\
\newline
Radix Complement\\
This is the most convenient way of representing a number.
The formula can be written as
\[
	N = \left(\sum_{i=0}^{digit\,length}\left(R-1\right) - i \right)\,+\,1
\]\label{rc}
where \textit{N} is the number's complement and \textit{i} is the index of the bit.\\ \\
Radix Reduced Complement\\
This one as the name suggests is similar to the \ref{rc} Equation but the final addition operation is not present in this one, this one suffers from the 0s representation, as 0 can have a set or unset sign bit.\\

\paragraph{}
In Radix \textit{r}, with the standard digit set \([0, r - 1]\), the number of digits needed to
represent the natural numbers in \([0, max]\) is \[
	\mathit{k}= \left\lfloor \log_{r}max\right\rfloor\, + 1 \,= \left\lceil log_{r}(max+1)\right\rceil
\]
With fixed-point representation using k whole and l fractional digits, We have
\[
	max = r^{k}-r^{-l}\,=r^{k} - ulp
\]
\textbf{"ulp"}stands for \textbf{unit in least significant position}, which \\For Integers \( ulp = 1 \).
\\ 
\newpage


\section{IEEE 754 floating point}
According to the standard a in 32bit machine a number with floating point is represented by:\textbf{ MSB is the sign bit}, \textbf{exponent}(8bit) is\[exponent +  \frac{2^{7}}{2}(bias)\] and, \textbf{23 bit for fraction} \autoref{float}
\begin{figure}[b!]
	\includegraphics{float.png}
	\caption{an example of 32bit IEEE 754}
	\label{float}
\end{figure}


