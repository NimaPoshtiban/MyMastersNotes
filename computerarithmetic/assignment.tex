\documentclass[16pt,answers]{exam}
\author{Nima Poshtiban}
\title{Assignments of Arithmetic}
\usepackage{datetime}
\usepackage{amsmath}
\usepackage[utf8]{inputenc}
\usepackage{hyperref}
\usepackage{color}
\newdate{date}{{26}{10}{2025}}
\date{\displaydate{date}}


\begin{document}
	\maketitle
	\tableofcontents
	\pagebreak
	\section{Assignment No.1}
\paragraph{}
\textbf{Unconventional radices}
\begin{questions}
\question Convert the negabinary number $\mathbf{(0001 1111 0010 1101)_{-two}}$ to radix 16
	(hexadecimal).\label{a}
	\begin{solution}[space]
		From the Radix we can elicit the base logic for solving this problem\newline
		\begin{enumerate}
			\item Assuming the initial index is 1, Odd digits represents positive values whereas even digits are the quite opposite.
			\item Applying the deduction, a mathematical formula has been extracted from the logic:
			\[
			  0001 1111 0010 1101 \Longrightarrow \sum_{i=1}^{16}{(d_{i})\times2^{i-1}\times(-1)^{i} }
			\]
			\item Calculating the initial value:
			\(-2^{0} + -2^{2} + -2^{3} + -2^{5}  + - 2^{8}  + -2^{9} + -2^{10} + -2^{11} + -2^{12} \Longrightarrow 2781_{decimal}\)
			\item Conversion from \(radix=10\) to \(radix=16\)
			doing the chain division and residue operations the result is \( ADD_{radix=16} \)
		\end{enumerate}
	\end{solution}
\question Repeat part \ref{a} for radix $\mathbf{-16}$ (negahexadecimal).
	\begin{solution}[space]
	do it!
\end{solution}
\question Derive a procedure for converting numbers from radix r to radix -r and vice
versa.
\end{questions}



\end{document}